\IEEEraisesectionheading{\section{Einführung}\label{sec:introduction}}
\IEEEPARstart{D}{er} Hype um den Modularisierungsansatz Microservices dauert an. Microservice-basierte Architekturen zeichnen sich durch viele Charakteristika aus. Neben der losen Kopplung und der moderaten Größe der Komponenten fokussiert sich dieser Ansatz vor allem auf eine erhöhte Flexibilität bei der Entwicklung und dem Deployment. Jede Komponente soll als eigene Einheit angesehen werden können, welche selbständig (weiter-) entwickelt und vor allem unabhängig von anderen Komponenten deployt werden kann. Durch Automatisierung wird die Software schnell und sicher über eine Continuous-Delivery-Pipeline in Produktion gebracht \cite{EberhardWolff.2015}. Allerdings hat Java EE in Bezug zu Microservices einen schlechten Ruf. Die agile Entwicklung und die kurzen Zyklen, bis ein Release ausgerollt werden kann, passen nicht zu den Spezifikationen des Enterprise-Umfelds. Java EE wird bevorzugt in großen Unternehmensanwendungen eingesetzt, welche oft komplexe Prozesse und unflexible Organisations- und Kommunikationsstrukturen enthalten \cite{jaxcenter.2016}. Dies widerspricht allerdings den Anforderungen an Microservices, welche durch ihren technologischen Ansatz sogar ein völliges Umstrukturieren der Architektur sowie Organisation nach sich ziehen können \cite{EberhardWolff.2015}. Obwohl Java EE ursprünglich für stark verteilte Anwendungen sowie fachlich orientierte Systeme ausgelegt ist und sämtliche Funktionalitäten bietet, welche zur Implementierung solcher Komponenten benötigt werden, gründeten namenhafte Anbieter und Organisationen MicroProfile.\\ \\
Durch die Definition einer Enterprise Java-Spezifikation, welche Microservice-Muster adressieren soll, wird eine Optimierung von Enterprise Java für Microservice-basierte Architekturen versprochen \cite{Microprofile.2017}. Diese Arbeit beschäftigt sich mit der Frage, warum überhaupt eine Optimierung gebraucht wird und warum Java EE tatsächlich kein geeignetes Werkzeug für diesen Architekturansatz ist. Dabei wird zuerst untersucht, wie Java EE die priorisierten Eigenschaften von Microservices umsetzt und welche Probleme es dabei gibt. Anhand dieser Probleme wird erläutert, warum MicroProfile ins Leben gerufen wurde und was dieser Ansatz an Enterprise Java verbessert. \\ \\
Da sich MicroProfile noch in einem frühen Entwicklungsstadium befindet, gibt es nur eine limitierte Auswahl an Literatur. Einen Überblick zum Thema gibt die offizielle Webseite microprofile.io, welches hier auch als Standard-Referenz herangezogen wird. Die Arbeit von Ueda et all beschäftigt sich unter anderem mit der Erstellung eines Benchmarks für ein Microservice basierend auf Java. Der Benchmark zeigt, dass ein Overhead zu beobachten ist. Aus dem Benchmark geht hervor, dass die Performance gedrosselt wird \cite{uht.2016}. Die Abhandlung stützt sich außerdem auf die Bücher Microservices \cite{EberhardWolff.2015} und Continuous Delivery \cite{EberhardWolff.2016} von Eberhard Wolff, welche Grundlagen bezüglich Microservices und die dafür erforderliche Automatisierung vermitteln. Das hier beschriebene Vorgehen zur Untersuchung der Technologie kann mit der Arbeit von Lars Röwekamp verglichen werden \cite{LarsRowekamp.2016}, welcher mehrere Beiträge zu diesem Thema verfasst hat und auf das Problem von Java EE eingeht. 





%Die Arbeit lehnt sich dabei an die von Lars Röwekamp verfassten Artikel auf diversen Plattformen an. Durch seine Beiträge manifestierte er seine grenzenlose Inkompetenz. Scheiß auf dich und dein Copyright.
