\IEEEraisesectionheading{\section{Introduction}\label{sec:introduction}}
\IEEEPARstart{T}{his} demo file is intended to serve as a ``starter file''
for IEEE Computer Society journal papers produced under \LaTeX\ using
IEEEtran.cls version 1.8b and later. I wish you the best of success.\\ \\
In der Enterprise-Community hält sich hartnäckig das Gerücht, dass Java EE nicht wirklich als Werkzeug für die neue Wunderwelt der Microservices geeignet sei. Die Wurzeln des Enterprise-Java-Standards sind jedoch genau dort zu finden, wo wir heute mit dem Architekturansatz der Microservices hin wollen – in stark verteilten, fachlich orientierten Systemen. Warum also hat Java EE, in Bezug auf Microservices, einen so schlechten Ruf? Und ist er gerechtfertigt? \\ \\
Unabhängig von den vielen Charakteristika, die eine Microservice-basierte Architektur im Detail auszeichnet, stehen vor allem die erhöhte Flexibilität bei Entwicklung und Deployment im Fokus. Etwas managementtauglicher könnte man es auch als „Time-to-Market-Optimierung“ bezeichnen. Leider scheint genau dieses „Time-to-Market“, selbst im Zeitalter von agiler Entwicklung und deutlich kürzeren Releasezyklen als noch vor Jahren, nicht wirklich zu Java EE zu passen. Dabei hebt bereits die erste J2EE-Spezifikation vor mehr als zehn Jahren hervor, dass Enterprise Developer sich zukünftig – dank J2EE – voll und ganz auf die Fachlichkeit der umzusetzenden Anwendung konzentrieren können, anstatt ihre Zeit mit der Lösung von Infrastrukturproblemen zu vergeuden. Zum ersten Mal rückt damals der Aspekt „Time to Market“ als Wettbewerbsvorteil für Anwender des Java-EE-Frameworks in den Fokus. Und auch der Begriff Enterprise Services taucht in der Spezifikation mehrfach auf: „The Java 2 Platform, Enterprise Edition reduces the cost and complexity of developing multitier, enterprise services. J2EE applications can be rapidly deployed and easily enhanced as the enterprise responds to competitive pressures.“ \\ \\
Woran liegt es also, dass Java EE den damaligen Ansprüchen nicht gerecht werden konnte und man sich als Java-EE-Entwickler im Kreise der Microservices-Fans wie ein Elefant im Porzellanladen fühlt? [1].\\ \\
Zu Beginn des Jahres 2016 verlangsamte sich das Tempo der Java EE-Veröffentlichungen, während sich die Branche schnell auf eine Microservices-Architektur zubewegte. Während dies geschah, war die Java-EE-Community fragmentiert, was Microservicemuster angeht und inkompatible Microservice-orientierte Laufzeiten lieferte. Um diese Kadenzfehlanpassung und -fragmentierung zu lösen, gründete eine Sammlung von Anbietern, Organisationen und Einzelpersonen MicroProfile, um Enterprise Java-Spezifikationen zu definieren, die Microservices-Muster adressieren.\\ \\
MicroProfile verwendet einen "Open Source" -Ansatz für die Entwicklung von Spezifikationen, indem es öffentlich als Gemeinschaft arbeitet und schnell Spezifikationen durchläuft, wodurch die Implementierung einer Spezifikation in Entwickler viel schneller erfolgt als bei einer traditionellen Standardorganisation. Wenn eine Spezifikation jedoch ausreichend ausgereift ist, soll sie zu einem Standard werden.
Innerhalb eines Zeitraums von ungefähr 15 Monaten hat MicroProfile 3 Veröffentlichungen erhalten und ist der Eclipse Foundation beigetreten, um seinem Ziel des schnellen, iterativen Fortschritts gerecht zu werden! Was folgt, ist ein kurzer Rückblick auf das, was wir im Detail geliefert haben. [2] \\ \\


(Muss germerged werden) Java EE bietet sämtliche Funktionalitäten wie beispielweise Health Check, Metrics oder Tolerance, an, welche zur Implementierung Restful-Microservies benötigt werden. Die Herausforderung besteht nicht aus der Umsetzung der Servicelogik, sondern aus der Implementierung der Interaktion aller Services. An dieser Stelle werden die Schwächen von JAVA EE bemerkbar. Java zielt darauf ab, dass Artefakte innerhalb eines Applikationsserver deployt werden. In diesem Fall repräsentiert ein Artefakt einen Microservice. Dies sollte erfolgen, da sonst die übergreifenden Dienste, wie zum Beispiel Konfiguration, Monitoring und Security übernehmen werden kann. Sollte der Applikationsserver fehlen oder existieren mehrere autonome, die wiederum ihren Microservice administrieren, gibt es keine übergeordnete Schaltzentrale. Aufgrund dieser Basis wird eine Initiative gesucht, die sich dieser Herausforderung stellt: Microprofile. Dieser Ansatz soll die Anforderung von Microservices berücksichtigen und deren mit Java EE ermöglichen. ‚Egal ob Health Check, Metrics, Fault Tolerance, JWT Propagation, Configuration, Tracing oder Open API, MicroProfile scheint die richtigen Antworten – sprich APIs – im Gepäck zu haben‘. Auf den ersten Blick scheint Microprofile das richtige zu sein