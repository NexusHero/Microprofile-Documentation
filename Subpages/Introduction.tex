\IEEEraisesectionheading{\section{Einführung}\label{sec:introduction}}
\IEEEPARstart{D}{er} Hype um den Modularisierungsansatz Microservices dauert an. Microservice-basierte Architekturen zeichnen sich durch viele Charakteristika aus. Neben der losen Kopplung und der moderaten Größe der Komponenten fokussiert sich dieser Ansatz vor allem auf eine erhöhte Flexibilität bei der Entwicklung und dem Deployment. Jede Komponente soll als eigene Einheit angesehen werden können, welche selbständig (weiter-) entwickelt und vor allem unabhängig von anderen Komponenten deployt werden kann. Durch Automatisierung soll die Software schnell und sicher über eine Continuous-Delivery-Pipeline in Produktion gebracht werden \cite{EberhardWolff.2015}. Allerdings hat Java EE in Bezug auf Microservices einen schlechten Ruf. Die agile Entwicklung und die kurzen Zyklen, bis ein Release ausgerollt werden kann, passen nicht zu den Spezifikationen des Enterprise-Umfelds. Java EE wird bevorzugt in großen Unternehmensanwendungen eingesetzt, welche oft komplexe Prozesse und unflexible Organisations- und Kommunikationsstrukturen enthalten \cite{jaxcenter.2016}. Dies widerspricht allerdings den Anforderungen an Microservices, welche durch ihren technologischen Ansatz sogar ein völliges Umstrukturieren der Architektur sowie Organisation nach sich ziehen können \cite{EberhardWolff.2015}. Obwohl Java EE ursprünglich für stark verteilte Anwendungen sowie fachlich orientierte Systeme ausgelegt wurde und es sämtliche Funktionalitäten bietet, welche zur Implementierung solcher Komponenten benötigt werden, gründeten eine Sammlung verschiedener namenhafte Anbieter und Organisationen MicroProfile.\\
Durch die Definition einer Enterprise Java-Spezifikation, welche Microservice-Muster adressieren soll, wird eine Optimierung von Enterprise Java für Microservice-basierte Architekturen versprochen \cite{Microprofile.2017}. Diese wissenschaftliche Arbeit beschäftigt sich mit der Frage, warum überhaupt eine Optimierung gebraucht wird und warum Java EE tatsächlich kein geeignetes Werkzeug für diesen Architekturansatz ist. Dabei wird zuerst untersucht, wie Java EE die priorisierten Eigenschaften von Microservices umsetzt und welche Probleme es dabei gibt. Anhand dieser Probleme wird erläutert, warum MicroProfile ins Leben gerufen wurde und was dieser Ansatz an Enterprise Java verbessert. \\
Die Arbeit von Ueda et all beschäftigt sich unter anderem mit der Erstellung eines Benchmarks für ein Microservice basierend auf Java. Der Benchmark zeigt, dass ein Overhead  zu beobachten ist. Aus dem Benchmark geht hervor, dass die Performance  gedrosselt wird \cite{uht.2016}. Dieses Papier stützt sich auf die Bücher Microservices \cite{EberhardWolff.2015} und Continuous Delivery \cite{EberhardWolff.2016} von Eberhard Wolff, welche umfangreiche Grundlagen bezüglich Microservices vermitteln. Das hier beschriebene Vorgehen zur Untersuchung der Technologie kann mit den Beiträgen von Lars Röwekamp auf jaxenter.de verglichen werden. In dessen Beitrag wird auf das Problem von Java EE eingegangen. 
