\IEEEraisesectionheading{\section{Introduction}\label{sec:introduction}}
\IEEEPARstart{M}{icroservice-basierte} Architekturen zeichnen sich durch viele Charakteristika aus. Neben der losen Kopplung und Verteilung der Dienste fokussiert sich dieser Ansatz vor allem auf eine erhöhte Flexibilität bei der Entwicklung und dem Deployment. Jede Komponente soll als eigene Einheit angesehen werden, welche selbständig (weiter-) entwickelt und vor allem unabhängig deployt werden kann. Durch Automatisierung soll die Software schnell und sicher in Produktion gebracht werden. Allerdings hat Java EE in Bezug auf Microservices einen schlechten Ruf, da die agile Entwicklung und kurzen Zyklen, bis ein Release ausgerollt wird, nicht in die Spezifikation des Enterprise-Umfelds passen. Java EE wird bevorzugt in großen Projekten eingesetzt, welche oft komplexe Prozesse und unflexible Organisations- und Kommunikationsstrukturen enthalten. Dies widerspricht allerdings den Anforderungen an Microservices [1]. MicroProfile verspricht allerdings eine Optimierung von Enterprise Java für Microservice-basierte Architekturen.
Java EE ist allerdings für stark verteilte Anwendungen sowie fachlich orientierte Systeme ausgelegt und bietet sämtliche Funktionalitäten an, welche zur Implementierung solcher Komponenten benötigt werden. Diese wissenschaftliche Arbeit beschäftigt sich mit der Frage, warum überhaupt eine Optimierung gebraucht wird und warum Java EE kein geeignetes Werkzeug für diesen Architekturansatz ist. 
