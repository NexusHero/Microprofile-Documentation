\section{Fazit und Ausblick}
Wie aus diesem Artikel hervorgeht, bietet Java EE bereits alles, was für die Entwicklung von Microservices benötigt wird. Zumindest, was den technologischen Aspekt angeht. Es können qualitativ hochwertige, fachlich orientierte Microservices mit Java EE implementiert werden. Allerdings ist die reine Entwicklung nicht das Problem. Die Automatisierung sämtlicher Phasen, welche bei Microservices in Verbindung mit Continuous Delivery verfolgt wird, kann mit Java EE nicht ausreichend umgesetzt werden. Aufgrund des schwergewichtigen Application Server, weist die Runtime schwächen auf. Die von Microservices angestrebte Skalierbarkeit ist somit nicht gegeben und das kontinuierliche Deployment von mehreren tausenden Instanzen ist nicht möglich. Erst wenn der Server auf die nötigsten Komponenten reduziert wird, kann eine entsprechende Automatisierung mit der gewünschten Effizienz realisiert werden. Durch diesen Ansatz wird ein Self-contained-Systems erreicht, die einige Parallelen zu dem Microservice-Ansatz aufweisen, allerdings noch zu groß und grobgranular sind. Viele Anbieter wie auch die Gründer von MicroProfile arbeiten daher an Lösungen, mit denen Java EE auf die für den Service benötigten Bestandteile reduziert wird. Somit können schlanke und schnell deploybare Microservices auf Grundlage von Enterprise Java konzipiert und implementiert werden. Somit können Teams mit umfangreicher Erfahrung weiter Anwendungen auf Basis von Java EE entwickeln. \\ \\
Die Schöpfer von MicroProfile wollen hier einen neuen Standard etablieren. Ihre erste Veröffentlichung (MicroProfile 1.0) umfasste daher lediglich JAX-RS 2.0, CDI 1.2 und JSON-P 1.0. Für die Entwicklung der Spezifikationen wird ein Open Source-Ansatz verfolgt. Unter Einbezug der Community werden gemeinsam Features entwickelt und geplant. Auf der offiziellen Webseite von www.microprofile.io werden Interessierte dazu aufgefordert über verschiedene Diskussionen in der MicroProfile Google Group, bei der Optimierung von Enterprise Java für Microservices zu helfen. Durch diesen Ansatz wird die Entwicklung beschleunigt und ist eine Spezifikation ausreichend ausgereift, wird das Ergebnis zur Standardisierung in Betracht gezogen. \\ \\
Die Zukunft von MicroProfile bleibt höchst spannend. Aktuell wird das Framework weiter ausgebaut, sodass die Diskrepanz zwischen Java-EE und den verschiedenen Microservice-Frameworks weiter verringert wird. Vor allem da Microservices gerade eine besondere Aufmerksamkeit genießen, steigt die Notwendigkeit für MicroProfile weiter an \cite{LarsRowekamp.2018}. Mit dieser Initiative kann Java EE auch ernsthaft im Microservice-Umfeld etabliert werden. Das nächste Update MicroProfile 2.0 wird bereits März 2018 veröffentlicht. Die neuen Features sind in Abbildung \ref{fig:features2.0} aufgeführt. 

\begin{figure}[h!]
	\centering
	\includegraphics[width=1.0\linewidth]{images/Microprofile20}
	\caption{Microprofile-2.0-Features \cite{Microprofile.2017}} %Generelle
	\label{fig:features2.0}
\end{figure}

% Can use something like this to put references on a page
% by themselves when using endfloat and the captionsoff option.
\ifCLASSOPTIONcaptionsoff
  \newpage
\fi