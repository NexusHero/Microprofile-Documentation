\section{Conclusion}
Fazit voerst nur aus dem Artikel kopiert!!\\ \\
Java EE bietet aus rein technologischer Sicht alles, was es zur Entwicklung von Microservices braucht. Trotzdem ist der Enterprise-Java-Standard für viele Microservice-Neueinsteiger nicht unbedingt die nächstliegende Wahl. Neben der eigentlichen Entwicklung ist vor allem die vollständige und effiziente Automatisierung sämtlicher Phasen des Software Development Lifecycles für eine erfolgreiche Einführung von Microservices von Belang. Und genau hier haben Java EE und die zugehörige Runtime klare Schwächen. Der Application Server ist einfach zu schwergewichtig, als dass tausende Instanzen permanent neu deployt werden könnten. Dies gilt nicht nur für die traditionsbehafteten Dinos am Markt, sondern auch für die auf das Web Profile ausgerichteten Leichtgewichte.\\ \\
Die notwendige Automatisierung ist mit der gewünschten Effizienz nur dann realistisch, wenn die Server noch weiter abspecken. Nach dem Vorbild von Dropwizard und Spring Boot tauchen in den letzten Monaten daher vermehrt Lösungen auf, mit denen sich die eigene Java-EE-Anwendung oder der Java-EE-basierte Microservice mit genau den Bestandteilen bootstrapen lässt, die für den Service benötigt werden. Das Resultat sind schlanke, schnell deploybare Microservices auf Basis von Java EE. Dank guter Integration von im Microservice-Umfeld etablierten Open-Source-Lösungen à la Netflix OSS und Co. kommen dabei auch das Management und Monitoring der Microservices nicht zu kurz – sowohl auf dem eigenen Server als auch in der Cloud.\\ \\
Es ist also mit Java EE durchaus möglich, neue Features in Form von fachlich orientierten Microservices mit hoher Qualität zu implementieren und schnell an den Markt zu bringen. Time-to-Market und Java EE müssen sich, richtig angegangen, nicht widersprechen, ganz im Gegenteil. Dank Standard kann auf jahrelang aufgebautes Fachwissen zurückgegriffen werden, zumindest dann, wenn die Java-EE-Entwickler bereit sind, den einen oder anderen alten Zopf abzuschneiden und sich auf neue, spannende Welten einzulassen. Denn, wie bereits zu Anfang angedeutet: Nicht Java EE ist das Problem, sondern die Zeit, in der das Framework groß geworden ist.



\appendices
\section{Proof of the First Zonklar Equation}
Appendix one text goes here.

% you can choose not to have a title for an appendix
% if you want by leaving the argument blank
\section{}
Appendix two text goes here.


% Can use something like this to put references on a page
% by themselves when using endfloat and the captionsoff option.
\ifCLASSOPTIONcaptionsoff
  \newpage
\fi