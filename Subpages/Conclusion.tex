\section{Conclusion}
Fazit vorerst nur kopiert!!\\ \\
Java EE bietet aus rein technologischer Sicht alles, was es zur Entwicklung von Microservices braucht. Trotzdem ist der Enterprise-Java-Standard für viele Microservice-Neueinsteiger nicht unbedingt die nächstliegende Wahl. Neben der eigentlichen Entwicklung ist vor allem die vollständige und effiziente Automatisierung sämtlicher Phasen des Software Development Lifecycles für eine erfolgreiche Einführung von Microservices von Belang. Und genau hier haben Java EE und die zugehörige Runtime klare Schwächen. Der Application Server ist einfach zu schwergewichtig, als dass tausende Instanzen permanent neu deployt werden könnten. Dies gilt nicht nur für die traditionsbehafteten Dinos am Markt, sondern auch für die auf das Web Profile ausgerichteten Leichtgewichte.\\ \\
Die notwendige Automatisierung ist mit der gewünschten Effizienz nur dann realistisch, wenn die Server noch weiter abspecken. Nach dem Vorbild von Dropwizard und Spring Boot tauchen in den letzten Monaten daher vermehrt Lösungen auf, mit denen sich die eigene Java-EE-Anwendung oder der Java-EE-basierte Microservice mit genau den Bestandteilen bootstrapen lässt, die für den Service benötigt werden. Das Resultat sind schlanke, schnell deploybare Microservices auf Basis von Java EE. Dank guter Integration von im Microservice-Umfeld etablierten Open-Source-Lösungen à la Netflix OSS und Co. kommen dabei auch das Management und Monitoring der Microservices nicht zu kurz – sowohl auf dem eigenen Server als auch in der Cloud. \\ \\
Es ist also mit Java EE durchaus möglich, neue Features in Form von fachlich orientierten Microservices mit hoher Qualität zu implementieren und schnell an den Markt zu bringen. Time-to-Market und Java EE müssen sich, richtig angegangen, nicht widersprechen, ganz im Gegenteil. Dank Standard kann auf jahrelang aufgebautes Fachwissen zurückgegriffen werden, zumindest dann, wenn die Java-EE-Entwickler bereit sind, den einen oder anderen alten Zopf abzuschneiden und sich auf neue, spannende Welten einzulassen. Denn, wie bereits zu Anfang angedeutet: Nicht Java EE ist das Problem, sondern die Zeit, in der das Framework groß geworden ist.\\ \\
Ein eigener JSR und eine damit verbundene Überführung des MicroProfile in den Java-EE-Standard ist zwar durchaus eine wichtige Option, steht aber nicht im Fokus der Bemühungen. Stattdessen geht es vielmehr darum, Einigkeit der involvierten Player zu signalisieren und gemeinsam mit der Community einen De-facto-Standard zu generieren. Gemeinsam mit der Community? Genau! Dass man das ganze Thema nicht nur aus Sicht der Application-Server-Hersteller betrachtet, sondern durchaus an der Meinung der Community interessiert ist, zeigt eine entsprechende Umfrage auf der Webseite von www.microprofile.io. Hier wird jeder aufgefordert, die aus seiner Sicht wichtigsten Aspekte eines Microservice (z. B. Start-up Time, Metrics, Disk Space, Circuit Breaker) sowie die für die Implementierung von Microservices sinnvollen Java-EE-APIs zu nennen. \\ \\
Mit der Initiative microprofile.io und dem zugehörigen MicroProfile ist etwas entstanden, das eine reale Chance darauf hat, Java EE im Umfeld von Microservices zu etablieren. Glaubt man den bisherigen Ankündigungen, haben die Initiatoren es verstanden, das Beste aus Standard und Community-driven in einem Ansatz zu vereinen: Herstellerunabhängigkeit und damit verbunden Portabilität bei gleichzeitig schnellen Reaktionszeiten.\\ \\
Wichtig zu verstehen ist, dass die Initiatoren von www.microprofile.io nicht den Glauben in Java EE verloren haben. Ganz im Gegenteil, Red Hat hat erst vor Kurzem noch einmal sein weiteres Engagement im Umfeld von Java EE 8 betont. Eine spätere Überführung des MicroProfiles in den Java-EE-Standard ist also durchaus denkbar und gewünscht. Wichtig ist aber auch, dass zukünftig alternative Wege im Enterprise-Java-Umfeld eine wichtige Rolle spielen werden, die sich mit hoher Wahrscheinlichkeit jenseits des Java Community Process in seiner derzeitigen Form bewegen.\\ \\
Jeder ist aufgefordert, diese alternativen Wege möglichst aktiv und intensiv mitzugestalten. Im konkreten Fall kann man dies schon heute über die MicroProfile Google Group. In diesem Sinne, zum Ende ausnahmsweise einmal nicht Stay tuned, sondern: Participate!
Mit Microprofile können Microservices auf Grundlage von Java konzipiert und implementiert werden. Aktuell wird das Framework weiterausgebaut, sodass die Diskrepanz zwischen Java-Ee und den verschiedenen Microservices-Framework-Ansätzen kompensieren. Das nächste Update Microprofile 2.0 kommt bereits im März 2018 mit den Features in Abbildung \ref{fig:features2.0}. Mit dem Hype um Microservices wird die Notwendigkeit für Microprofile noch weiter ansteigen \cite{LarsRowekamp.2018}.

\begin{figure}[h!]
	\centering
	\includegraphics[width=1.0\linewidth]{images/Microprofile20}
	\caption{Microprofile-2.0-Features \cite{Microprofile.2017}} %Generelle
	\label{fig:features2.0}
\end{figure}

% Can use something like this to put references on a page
% by themselves when using endfloat and the captionsoff option.
\ifCLASSOPTIONcaptionsoff
  \newpage
\fi