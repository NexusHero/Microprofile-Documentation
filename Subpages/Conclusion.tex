\section{Conclusion}
Wie aus diesem Artikel hervorgeht, bietet Java EE bereits alles, was für die Entwicklung von Microservices benötigt wird. Zumindest, was den technologischen Aspekt angeht. Es können auch qualitativ hochwertige, fachlich orientierte Microservices mit Java EE implementiert werden. Allerdings ist die reine Entwicklung nicht das Problem. Die Automatisierung sämtlicher Phasen, welche bei Microservices in Verbindung mit Continuous Delivery verfolgt wird, kann mit Java EE nicht ausreichend umgesetzt werden. Aufgrund des schwergewichtigen Application Server, weist die Runtime schwächen auf. Es ist schlicht nicht möglich, zigtausend Instanzen kontinuierlich neu zu deployen. Erst wenn der Server nur auf die nötigsten Komponenten reduziert wird, wird eine entsprechende Automatisierung mit der gewünschten Effizienz realistisch. Zwar erreicht man durch diesen Ansatz eher Self-contained Systems, die zwar einige Parallelen zu dem Microservice-Ansatz aufweisen, allerdings noch zu groß und grobgranular sind. Viele Anbieter wie auch die Gründer von MircoProfile arbeiten daher an Lösungen, mit denen Java EE auf die für den Service benötigten Bestandteile reduziert wird. Somit können auch schlanke und schnell deploybare Microservices auf Grundlage von Enterprise Java konzipiert und implementiert werden.\\ \\
Die Schöpfer von MicroProfile wollen hier einen neuen Standard etablieren. Ihre erste Veröffentlichung enthielt daher lediglich (…). \\ \\
Für die Entwicklung der Spezifikationen wird ein Open Source-Ansatz verfolgt. Man arbeitet öffentlich als Gemeinschaft unter Einbezug der Community. Auf der offiziellen Webseite von www.microprofile.io werden Interessierte dazu aufgefordert über verschiedene Diskussionen in der MicroProfile Google Group, bei der Optimierung von Enterprise Java für Microservices zu helfen. Durch diesen Ansatz wird die Entwicklung beschleunigt und ist eine Spezifikation ausreichend ausgereift, wird sie zum Standard erklärt [2].  \\ \\
Die Zukunft von MicroProfile bleibt höchst spannend. Aktuell wird das Framework weiter ausgebaut, sodass die Diskrepanz zwischen Java-EE und den verschiedenen Microservice-Frameworks weiter verringert wird. Vor allem da Microservices gerade eine besondere Aufmerksamkeit genießen, steigt die Notwendigkeit für MicroProfile weiter an \cite{LarsRowekamp.2018}. Mit dieser Initiative kann Java EE auch ernsthaft im Microservice-Umfeld etabliert werden. Das nächste Update MicroProfile 2.0 wird bereits März 2018 veröffentlicht. Die neuen Features sind in Abbildung \ref{fig:features2.0} aufgeführt. 

\begin{figure}[h!]
	\centering
	\includegraphics[width=1.0\linewidth]{images/Microprofile20}
	\caption{Microprofile-2.0-Features \cite{Microprofile.2017}} %Generelle
	\label{fig:features2.0}
\end{figure}

% Can use something like this to put references on a page
% by themselves when using endfloat and the captionsoff option.
\ifCLASSOPTIONcaptionsoff
  \newpage
\fi