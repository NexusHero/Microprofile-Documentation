% for Computer Society papers, we must declare the abstract and index terms
% PRIOR to the title within the \IEEEtitleabstractindextext IEEEtran
% command as these need to go into the title area created by \maketitle.
% As a general rule, do not put math, special symbols or citations
% in the abstract or keywords.
\IEEEtitleabstractindextext{%
	\begin{abstract}
		Dieser Artikel beschäftigt sich mit dem Einsatz von Java-EE in Microservices-Architekturen. Dabei bietet sie alle Features an, welche zur Implementierung von Microservices benötigt werden. Aus Sicht der angebotenen Funktionalitäten stellt Java-EE keine Hürde dar. Jedoch ist die Automatisierung des Build- und Deployment-Vorgangs ein langfristiges Problem. Dies ist auf den erzeugten Overhead zurückzuführen. Aufgrund dessen stellt sich der Einsatz von Continuous Delivery, das eine essentielle Anforderung an Microservices-Architekturen darstellt, als schwierig heraus. Als Lösung für dieses Problem wird Microprofile in Betracht gezogen, welches eine Umsetzung von Microservices auf Basis von leichtgewichtigen Java-EE-Funktionalitäten ermöglichen soll. Das Framework bietet bereits einige Funktionen wie beispielsweise HealthCheck an. Sie befindet sich noch im Anfangsstadium und deswegen sind nicht alle benötigten Features implementiert.
	\end{abstract}
	
	% Note that keywords are not normally used for peerreview papers.
	\begin{IEEEkeywords}
		Microprofile, Java-EE, Microservices , API, Version 1.3
\end{IEEEkeywords}}