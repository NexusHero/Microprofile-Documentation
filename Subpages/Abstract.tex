% for Computer Society papers, we must declare the abstract and index terms
% PRIOR to the title within the \IEEEtitleabstractindextext IEEEtran
% command as these need to go into the title area created by \maketitle.
% As a general rule, do not put math, special symbols or citations
% in the abstract or keywords.
\IEEEtitleabstractindextext{%
	\begin{abstract}
		Dieser Artikel beschäftigt sich mit dem Einsatz von Java-EE in Microservices-Umfeld. Dabei bietet Java-EE alle benötigten Features an, sodass Microservices auf Basis von Java implementiert werden können. Es stellt sich heraus, dass Java-EE aus Sicht der angebotenen Funktionalitäten kein Hindernis darstellt. Lediglich die Automatisierung des Build- und Deployment-Vorgangs stellt hier ein langfristiges Problem dar. Dies ist auf den erzeugten Overhead beim Deployment zurückzuführen. Aufgrund dessen stellt sich der Einsatz von Continuous Delivery als schwierig heraus, da dies eine essentielle Anforderung von Microservices darstellt. Als Lösung für dieses Problem wird Microprofile in Betracht gezogen, welches eine Umsetzung von leichtgewichtigen Microservices ermöglichen soll. Das Framwork befindet sich  im Anfangsstadium und es sind nicht alle benötigten Features implementiert.
	\end{abstract}
	
	% Note that keywords are not normally used for peerreview papers.
	\begin{IEEEkeywords}
		Computer Society, IEEE, IEEEtran, journal, \LaTeX, paper, template.
\end{IEEEkeywords}}