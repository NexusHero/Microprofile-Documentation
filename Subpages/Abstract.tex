% for Computer Society papers, we must declare the abstract and index terms
% PRIOR to the title within the \IEEEtitleabstractindextext IEEEtran
% command as these need to go into the title area created by \maketitle.
% As a general rule, do not put math, special symbols or citations
% in the abstract or keywords.
\IEEEtitleabstractindextext{%
	\begin{abstract}
		Dieser Artikel beschäftigt sich mit MicroProfile, der Optimierung von Java Enterprise für Microservice-basierte Anwendungen. Die Arbeit setzt sich mit der Forschungsfrage auseinander, warum Java EE in diesem Umfeld eine Optimierung benötigt. Aus technologischer Sicht stellt es alle notwendigen Funktionen bereit. Dabei wird untersucht, welche negativen Aspekte sich hier unter der Verwendung von Java EE ergeben. Aus dieser Untersuchung geht hervor, dass nicht die Entwicklung der kleinen, unabhängigen Dienste das Problem darstellt, sondern die dazu notwendige Automatisierung des Deployments. Continuous Delivery ist eine essentielle Disziplin im Microservice-Umfeld. Für die hier angestrebten Anforderungen ist der Overhead von Java EE allerdings zu groß. Es werden einige Ansätze diskutiert, die eine potentielle Verbesserung mit sich bringen. Hierbei wird MicroProfile herangezogen, welches auf Basis von leichtgewichtigen Java-EE-Funktionalitäten, eine Lösung des Problems darstellt.		 
	\end{abstract}
	
	% Note that keywords are not normally used for peerreview papers.
	\begin{IEEEkeywords}
		Microprofile, Java-EE, Microservices , API, Version 1.3
\end{IEEEkeywords}}