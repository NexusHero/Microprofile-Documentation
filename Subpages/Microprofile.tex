\section{MicroProfile}
Wie aus den obigen Kapiteln hervorging, erfolgen Optimierungen bisher durch triviale Ansätze. Ist der Anwendungsserver zu groß, werden nur die für den Microservice zwingend benötigten Komponenten mit eingebunden. Allerdings weisen Microservices, die auf dieser Basis aufbauen noch immer Schwachstellen auf. MicroProfile verspricht hier eine Optimierung von Enterprise Java für Microservice-Architekturen. Sie bietet Java EE-Entwickler die Möglichkeit, ihr bestehendes Wissen über Java weiterhin nutzen zu können \cite{GabrielaMotroc.2016}. Hinter dem Namen MicroProfile stecken einige große Unternehmen, die gemeinsam mit der Community an dieser Lösung arbeiten, um den aktuellen Marktbedürfnissen entgegenzukommen. Es wird versucht einige vorhandene Tools zu nutzen und diese mit neuen Werkzeugen zu verbinden. Dadurch soll eine Basisplattform geschaffen werden, die dem Microservice-Ansatz genügt. Die initiale Version dieser Lösung stellte dabei einen minimalen Satz an APIs zur Verfügung. Aktuell werden die Funktionen angeboten, welche in Abbildung \ref{fig:features1.3} aufzeigt werden \cite{Microprofile.2017}.  \\ \\
\begin{figure}[h!]
	\includegraphics[width=1.0\linewidth]{images/Microprofile13}
	\caption{MicroProfile-1.3-Features \cite{Microprofile.2017}} %Generelle
	\label{fig:features1.3}
\end{figure}
\subsection{Config-API} 
Die Config-API trennt Anwendungslogik und Konfiguration. Dadurch kann ein Microservice dynamisch auf die Laufzeitumgebung angepasst werden. Bei diesem Ansatz stellt die Konfiguration von heterogenen Quellen in unterschiedlichen Formaten, wie beispielsweise Umgebungsvariablen und Datenbanken, eine Herausforderung dar. Bei statischen Konfigurationswerten genügt eine Initiierung beim Start des Prozesses. Bei dynamischen Konfigurationen hingegen sollten administrative Maßnahmen ergriffen werden. Ein Beispiel hierfür wäre das Prüfen auf Aktualität und Korrektheit der Konfigurationen. 
Diese Problematik greift Config-API auf. Die API ermöglicht eine Zugriffsvereinheitlichung auf unterschiedliche Konfigurationen. Standardwerte werden in einer Datei abgelegt. Diese können bedarfsgerecht geladen und in der Umgebung überschrieben werden. Auch das Einbinden von weiteren Quellen ist möglich.
Die Zugriffsverwaltung auf die Konfigurationen können auf zwei Wegen geschehen: ConfigProvider und ConfigBuilder. Um diese umsetzen zu können, benötigt es zuvor die Instanziierung der Config-Klasse. Der Builder erlaubt die individuelle Anpassung der Konfiguration. \\ \\ Bei dem ConfigProvider gilt es zu erwähnen, dass beim Aufruf der Methode getConfig(), die zurückgelieferte Konfigurationsinstanz aus Effizienzgründen gecached wird. Dies müsste bei paralleler Programmierung beachtet werden. Der Zugriff kann auch über CDI-Injection-Annotationen geschehen. Sobald ein Konfigurationswert durch den Schlüssel nicht gefunden wird, reagiert das System mit einer NoSuchElement-Exception. Bei CDI-Injection wird Deployment-Exception geworfen. Auch optionale Konfiguration kann durch die Methode getOptionalValue().orelse() in Betracht gezogen werden, sodass im Zweifelsfall eine andere Konfiguration angezogen wird \cite{LarsRowekamp.2017}. 

\subsubsection{Just-In-Time-Konfiguration} 
Damit ein Microservice bei Konfigurationsänderung nicht neu gestartet werden muss, bietet die Config-API ein Mechanismus, sodass Konfigurationswerte dynamisch zur Laufzeit geladen werden können. Um den aktuellen Konfigurationswert zu erhalten und nicht den Wert zum Zeitpunkt der Injection, sollte ein Provider injiziert werden. Dadurch wird der aktuelle Wert angezogen. Just-in-Time steht in diesem Zusammenhang für die Aktualität des Konfigurationswertes. Der Entwickler ist für die Aktualisierung der Konfigurationswerte verantwortlich \cite{LarsRowekamp.2017}. 

\subsubsection{Converter}
Durch den Converter können die Konfigurationswerte, welche ausschließlich aus Strings besteht, in Java-Typen konvertiert, sodass auch andere Typen als Strings verwendet werden können. Bereits für einige Java-Typen gibt es den Build-in-Converter. Es werden Typen wie beispielsweise Boolean, Integer, Long, Double, URL und LocalDateTime unterstützt \cite{LarsRowekamp.2017}.

\subsection{HealthCheck-API} \label{healthcheck}
Die HealthCheck-API ermöglicht, dass der aktuelle Status eines Services abgefragt werden kann. Antworten beinhalten Schlüsselwörter wie 'Up' und 'Down'. Falls der Service nicht verfügbar sein sollte, können hier Maßnahmen, wie beispielsweise das Neustarten ergriffen werden \cite{LarsRowekamp.2017}. Des Weiteren wird durch die Annotation  Health eine automatische REST-API-Funktion zur Verfügung gestellt. Auf Basis dieser Funktion können Monitoring- und Management-Tools durch Rest-Calls automatisiert überprüfen, ob ein Neustart erforderlich ist \cite{LarsRowekamp.2017c}.  

\subsection{FaultTolerance-API} \label{faulttolerance}
Die FaultTolerance-API hat den Zweck, eine verbesserte Systemstabilität zu erreichen. Des Weiteren können Resilience-Patterns wie beispielsweise Fallback, Timeout oder Bulkhead auf Basis von Java-Annotationen umgesetzt werden. Aber auch die Ausprogrammierung dieser Patterns ist möglich. Dadurch erhöht sich die Verfügbarkeit eines Microservices \cite{ibm.2017}.

\subsection{Metrics}
Im Gegensatz zur HealthCheck-API wird die Metrics-API dazu verwendet, feingranulare Systeminformationen und kritische Systemparameter zur Laufzeit zu überwachen. Diese Überwachung läuft über Softwareagenten, wodurch Prognosen über künftige Serviceverhalten erstellt werden können. Durch Einsatz von Contexts und Dependency Injection kann diese Funktionalität verwendet werden. Auf diese Weise lässt sich zum Beispiel auf eine erhöhte Auslastung mit dem Start neuer Service-Instanzen oder proaktiv auf sich abzeichnende Engpässe reagieren \cite{LarsRowekamp.2017c}. 

\subsection{Json Web Token (JWT) Propagation}
Die JWT-Propagation bietet Authentifizierung und Autorisierung innerhalb eines Microservices an. Innerhalb eines JAX-RS-Containers können durch Annotationen auf Instanz des JWT sowie Claims zugegriffen werden. Hier lassen sich Java EE-Funktionalitäten wie  RolesAllowed gemeinsam mit JWT nutzen. Dies ist aufgrund von automatischem Mapping auf den Java EE SecurityContext möglich \cite{LarsRowekamp.2017c}. 


\subsection{OpenAPI und OpenTracing-API}
Durch OpenAPI können OpenAPI-V3-Dokumente aus JAX-RS-Applikationen erstellt und als unabhängige Beschreibungsformat für REST APIs genutzt werden. Diese API hat den Zweck, eine automatisierte Generierung von Microservices-APIs bereitzustellen. Das erlaubt die Verwaltung über API-Management-Werkzeugen.
OpenTracing-API versucht ein Standard bzgl. verteiltes Tracing zu setzen. Hierdurch können Requests nachverfolgt werden \cite{DominikMohilo2018}.  

