\section{Microprofile}

\subsection{Config-API} 

Die Confg API trennt Anwendungslogik und Konfiguration, sodass der Microservice dynamisch auf die Laufzeitumgebung angepasst werden kann. Dieser Ansatz verbirgt ein Herausforderungen: Konfigurationen wie zum Beispiel für Dateien, Umgebungsvariablen und Datenbank stammen aus heterogenen Quellen. Das führt auch dazu, dass die einzelnen Konfigurationen auch in unterschiedlichen Formaten vorfindbar sind, welches die Administration erschwert. Bei statischen Konfigurationswerten genügt eine Initiierung beim Start des Prozesses. Hingegen beim dynamischen Konfigurationen sollten administrative Maßnahmen ergriffen werden, welche zur Laufzeit erfolgen müssen, wie zum Beispiel das Prüfen auf Aktualität und Korrektheit der Konfigurationen. Diese Problematik soll die MicroProfile-Config-API aufgreifen und lösen. Sie ermöglicht die Zugriffvereinheitlichung auf unterschiedliche Konfigurationen, die wiederum priorisiert und dadurch gezielte Konfigurationsüberschreibung möglicht wird. Standardwerte werden in einer „microprofile-config.properties“ abgelegt .Diese können dann bedarfsgerecht angezogen und in der Umgebung überschrieben werden. Auch das Einbinden von weiteren Quellen ist möglich.
Die Zugriffsverwaltung auf die Konfigurationen können auf zwei Wegen geschehen: ConfigProvider und ConfigBuilder. Um diese umsetzen zu können benötigt es davor die Instanziierung der Config-Klasse. Der Builder erlaubt die individuelle Anpassung der Konfiguration und die Instanz wird nicht gecached. Beim ConfigProvider gilt es zu erwähnen, dass beim Aufruf der Methode getConfig(), die zurückgelieferte Konfigurations-Instanz - aus Effizienzgründen - gecached wird. Dies müsste bei paralleler Programmierung beachtet werden. Der Zugriff kann allerdings auch über CDI-Injection-Annotationen geschehen. Sobald ein Konfigurationswert durch den Schlüssel nicht gefunden wird, reagiert das System mit einer NoSuchElement-Exception. Bei CDI-Injection wird Deployment-Exception geworfen. Auch optionale Konfiguration kann durch die Methode getOptionalValue().orelse() in Betracht gezogen, sodass im Zweifelsfall eine andere Konfiguration angezogen werden kann. 

\subsection{Just-In-Time-Konfiguration} 
Damit ein Microservice bei Konfigurationsänderung nicht neugestartet werden muss, bietet die Config-API ein Mechanismus an, sodass Konfigurationswerte dynamisch zur Laufzeit geladen werden können. Um den aktuellen Konfigurationswert zu erhalten und nicht den Wert zum Zeitpunkt der Injection, sollte, muss ein Provider injiziert werden. Dadurch wird immer der aktuelle Wert angezogen. Just-in-Time steht in diesem Zusammenhang für die Aktualität des Konfigurationswertes. Die Aktualisierung der Werte innerhalb der Quelle und darauf bezogen das Refresh ist dem Autor der ConfigSource-Klasse überlassen. 

\subsection{Converter}
Durch den Converter können die Konfigurationswerte, welche ausschließlich aus Strings besteht, in Javatypen konvertiert, sodass auch andere Typen als Strings verwendet werden können. Bereits für einige Javatypen gibt es den Build-in-Converter. Es werden Typen, wie zum Beispiel Boolean, Integer, Long, Double, URL und LocalDateTime, unterstützt.
